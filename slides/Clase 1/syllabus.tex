\documentclass[letterpaper]{article}
\usepackage[utf8]{inputenc}
\usepackage[spanish]{babel}
\usepackage{amssymb}
\usepackage{tabularx}
\usepackage{amsthm}
\usepackage{amsmath}
\usepackage{graphicx}
\usepackage{epstopdf}
\usepackage{float}
\usepackage{amsfonts}
\usepackage{graphics}
\usepackage{graphpap}
\usepackage{latexsym}
\usepackage{makeidx}
\usepackage{enumerate}
\usepackage{rotating}
\usepackage{url}
\usepackage{multirow}
\usepackage{indentfirst}
%\usepackage[nofiglist,notablist,tablesfirst,nomarkers]{endfloat} % figuras al final
\usepackage[margin=3cm]{geometry}
\usepackage{hyperref}
\date{\vspace{-5ex}}
\begin{document}

\title{Análisis Cuantitativo I} 
\author{Facultad de Ciencias Humanas \\
	Universidad del Rosario}
\date{{\sc ii} semestre 2016}

\maketitle

\begin{tabular}{ll}
		{\bf Tipo de Asignatura:} & Obligatoria \\
			{\bf Tipo de saber:} & Obligatoria básica o
			de fundamentación \\
					{\bf Número de Créditos:} & 3 \\
	{\bf Profesor:} & Carlos Cardona Andrade \\
	{\bf Horario de clase:} & Viernes 7 - 10 am \\
	{\bf Salón:} &  Aula Virtual D  \\
	{\bf Correo:} & carlos.cardonaa@urosario.edu.co \\
	{\bf Horario de atenci\'on:} & Disponibilidad Flexible \\
\end{tabular}

\section{Resumen}
Este curso introduce a los estudiantes en el abordaje de los fenómenos sociales desde una óptica cuantitativa. Más específicamente busca brindar las bases de la estadística descriptiva aplicada a las ciencias sociales. A diferencia de un curso de estadística en abstracto, este curso busca incorporar el análisis estadístico a preguntas de investigación en ciencias sociales. Para ello la propuesta metodológica incluye el desarrollo de un proyecto cuantitativo a lo largo del curso que permita entender y aprender haciendo todo el proceso de una investigación cuantitativa. Este proyecto implicará pensar preguntas de investigación, hipótesis, variables, indicadores, medición, así como el diseño de un cuestionario y el armado de una base de datos que luego será analizada usando teoría y análisis sustantivo. Si bien no es pre-requisito formal, el curso está precedido lógicamente por un curso introductorio de métodos de investigación que pertenece al Ciclo Básico. Producir y analizar información cuantitativa constituye una herramienta clave para el desempeño de los egresados de Sociología en su vida profesional. En particular, aprender a diseñar, aplicar y analizar cuestionarios de encuestas es muy útil para hacer caracterizaciones de población que sirvan para diagnósticos o líneas de base, habilidades que nuestros egresados usan tanto en investigación básica como aplicada. 

\section{Propósitos de Formación del Curso}
\begin{itemize}
\item	Introducir y motivar al estudiante en la perspectiva cuantitativa de la investigación social. 
\item	Entrenar al estudiante en diseño de cuestionarios de encuestas y construcción de bases de datos. 
\item	Entrenar al estudiante en la generación de nuevas variables (recodificación y generación de índices).
\item	Entrenar a los estudiantes en el resumen de datos y en la presentación de los mismos (e.g., medidas de tendencia central y dispersión, gráficos descriptivos y tablas para análisis univariado y bivariado, etc.).
\item	Entrenar a los estudiantes en medidas de asociación bivariada para distintos tipos de variables. 
\item	Capacitar a los estudiantes en algún paquete estadístico (Excel, SPSS, STATA, etc.)
\item	Introducir regresión lineal (que será trabajada en Análisis cuantitativo II). 
\item	Ilustrar con ejemplos de artículos científicos cómo se usa la estadística descriptiva para responder a preguntas teóricas relevantes. 
\item	Este curso está dirigido a estudiantes de varias carreras incluyendo sociología, antropología, historia, filosofía, periodismo, artes liberales, ciencias políticas y relaciones internacionales. Se recomienda incluir ejemplos y ejercicios que respondan a esa diversidad disciplinaria, incluyendo temas y datos sociales y/o políticos. Igualmente, los trabajos durante el semestre y el trabajo final deben tocar temas de interés a las carreras de los estudiantes según sus carreras.
\end{itemize}

\section{Resultados de aprendizaje esperados (RAE)}

\begin{itemize}
\item	El estudiante puede formular preguntas cuantitativas e hipótesis, operacionalizar conceptos, diseñar su medición.
\item	El estudiante puede diseñar un cuestionario de encuesta, aplicarlo y armar una base de datos a partir de él.
\item	El estudiante es capaz de manipular una base de datos en un programa de análisis estadístico. 
\item	El estudiante es capaz de describir variables de interés y sus relaciones, así como de analizar resultados. 
\item	El estudiante puede comprender artículos científicos que usen estadística descriptiva. 

\end{itemize}

\section{Contenidos}
\begin{itemize}
	
\item	INTRODUCCIÓN AL DISEÑO Y ANÁLISIS CUANTITATIVOS. Preguntas cuantitativas. Hipótesis. Operacionalización. Medición. Tipos de variables e implicaciones para el análisis.
\item	RECOLECCIÓN DE DATOS. Conceptos básicos (variables, indicadores, etc.); encuestas (tipos, etc.); codificación (por ejemplo, usando la propia encuesta diseñada por el curso o el grupo); construcción de indicadores; población; introducción al muestreo.
\item	ESTADÍSTICA DESCRIPTIVA. Tabulación; medidas de tendencia central; medidas de dispersión; tablas y gráficos.
\item	ESTADÍSTICA INFERENCIAL. Probabilidad; curva normal; distribución muestral; teorema del límite central; Ley de los grandes números; intervalos de confianza; pruebas de hipótesis; prueba t; prueba Chi$^2$.  
\item	ANÁLISIS BIVARIADO. Covarianza y r de Pearson; regresión simple (intercepto, coeficiente beta y $R^2$).
\end{itemize}

\section{Evaluación}
\begin{enumerate}
	\item Proyecto de Análisis de Datos(30\%)
\item Examen Parcial (25\%)
\item Examen Final (25\%)
\item Talleres (10\%)
\item Quices (10\%)
\end{enumerate}

\section{Acuerdos de funcionamiento (Reglas de juego) }
\begin{enumerate}

	\item  {\bf Asistencia a clases} 
	\begin{enumerate}


\item Al acumular 4 fallas (10\% de las clases) la asignatura se pierde con 0.0. En todas las sesiones se
tomará lista al inicio de la clase. Tendrán falla los estudiantes que no asistan a clase, y quienes
se ausenten después de haberlos llamado.
	\end{enumerate}
	
	\item  {\bf Comportamiento en el aula} 
	\begin{enumerate}
		
\item Se requiere que los estudiantes mantengan apagados los equipos electrónicos (celulares,
computadores portátiles, MP3, Ipod, etc.).
	\end{enumerate}

	\item  {\bf Entrega de trabajos y exámenes} 
	\begin{enumerate}


\item Los trabajos asignados para la clase deben ser entregados en la fecha y hora establecidas por el
profesor.
\item Solo se recibirán trabajos de forma impresa; no se recibirán trabajos en formato electrónico por
vía email.
Los trabajos deberán ser originales; los casos de copia/plagio serán regulados a través de las
disposiciones de la Universidad del Rosario.
\item En caso de incumplimiento sobre alguna actividad de evaluación (trabajo, parcial, etc.),
solamente se aceptarán trabajos/reposiciones en las fechas diferentes a las señaladas, cuando el estudiante presente una excusa oficial y autorización tramitada en la Secretaría Académica de la ECH.
\item Los trabajos/parciales entregados al profesor serán calificados y retroalimentados dentro de los
plazos establecidos por el reglamento académico (dos semanas).
Los estudiantes tienen plazo de una semana para reclamar sus trabajos/parciales, y tienen
máximo una semana a partir de esa devolución, para hacer preguntas y/o reclamos.
\item Los estudiantes deben activar y revisar el MOODLE de la asignatura para acceder a los
materiales del curso, y para la comunicación entre los estudiantes y el profesor.
	\end{enumerate}
\end{enumerate}

\section{Bibliografía}
\subsection{Referencias Principales}
\begin{itemize}
\item Ferris Ritchey (2006): Estadística para las ciencias sociales, McGraw Hill.
\item Guillermo Briones (2003): Métodos y Técnicas de investigación para las Ciencias Sociales,
Trillas.
\item Earl Babbie (2000): Fundamentos de la Investigación Social, Thomson Editores.
\end{itemize}

\subsection{Referencias Adicionales}
\begin{itemize}
	\item Agresti, Alan and Finlay, Barbara. 1997. Statistical Methods for the Social Sciences, 3rd edition. Upper
	Saddle River, NJ: Prentice Hall.
	\item Kellstedt, P. M., \& Whitten, G. D. (2013). The Fundamentals of Political Science Research. Cambridge University Press.
\item Bertrand, Marianne and Mullainathan, Sendhil. (2004). “Are Emily and Greg More
Employable than Lakisha and Jamal?: A Field Experiment on Labor Market Discrimination.”
American Economic Review 94, 991–1013.
\item Wantchekon, L. (2003). “Clientelism and voting behavior: Evidence from a field experiment
in Benin.” World Politics 55, 399–422.

\item Posner, D. N. (2004). The political salience of cultural difference: Why Chewas and Tumbukas are allies in Zambia and adversaries in Malawi. American Political Science Review, 98(04), 529-545.
\item Bittman, M., England, P., Sayer, L., Folbre, N., \& Matheson, G. (2003). When Does Gender Trump Money? Bargaining and Time in Household Work. American Journal of Sociology, 109(1), 186-214.
\end{itemize}
\section{Actividades por Sesión}
\subsection{Sesión 1 (29 de julio)}
\begin{enumerate}
	\item Introducción al curso
	\item ¿Qué es la estadística?
	\item Tipos de Variables
	\item Introducción a Excel
\end{enumerate}

\subsection{Sesión 2 (5 de agosto)}
\begin{enumerate}
\item Diseños de Investigación
\item La medición
\item Medidas de Tendencia Central
\item Manejo de Excel
\begin{itemize}
	\item Ritchey (2006): Capítulo 4, p.107-126
\end{itemize}
\end{enumerate}

\subsection{Sesión 3 (12 de agosto)}
\begin{enumerate}
\item Medidas de dispersión
\item Ejercicios de tendencia central y dispersión
\begin{itemize}
\item Ritchey (2006):
Capítulo 5, p.
136-156
\end{itemize}
\end{enumerate}

\subsection{Sesión 4 (19 de agosto)}
\begin{enumerate}
\item Probabilidad 
\item La curva normal como distribución de probabilidad
\begin{itemize}
\item Ritchey (2006):
Capítulo 6, p.
168-195
\end{itemize}
\end{enumerate}
\subsection{Sesión 5 (26 de agosto)}
\begin{enumerate}
\item Introducción a Stata
\item Repaso
\item Entrega Taller 1
\end{enumerate}

\subsection{Sesión 6 (2 de septiembre)}
\begin{enumerate}
\item {\bf Examen Parcial}
\end{enumerate}

\subsection{Sesión 7 (9 de septiembre)}
\begin{enumerate}
\item Introducción a la
estadística inferencial
\item Teorema del Límite Central
\item Ley de los Grandes Números
\item Discusión acerca del tema del proyecto de investigación

\begin{itemize}
\item Ritchey (2006):
Capítulo 7, p.
206-224
\end{itemize}
\end{enumerate}

\subsection{Sesión 8 (23 de septiembre)}
\begin{enumerate}
\item Stata como herramienta estadística (análisis descriptivo)
\item Intervalos de confianza
\begin{itemize}
\item Ritchey (2006):
Capítulo 8, p.
237-259

\end{itemize}
\end{enumerate}

\subsection{Sesión 9 (30 de septiembre)}
\begin{enumerate}
\item Introducción a las pruebas de hipótesis
\item Gráficas en Stata
\begin{itemize}
\item Ritchey (2006): Capítulo 9, p.
267-304
\end{itemize}
\end{enumerate}
\subsection{Sesión 10 (14 de octubre)}

\begin{enumerate}
\item Prueba t
\item Entrega Taller 2
\begin{itemize}
\item Ritchey (2006):
Capítulo 11, p.
368-397
\end{itemize}
\end{enumerate}

\subsection{Sesión 11 (21 de octubre)}
\begin{enumerate}
\item Las distribuciones Chi$^2$ y Binomial
\item Pruebas de hipótesis en Stata
\begin{itemize}
\item Ritchey (2006):
Capítulo 13, p.
464-495
\end{itemize}
\end{enumerate}

\subsection{Sesión 12 (28 de octubre)}
\begin{enumerate}
	\item Correlación y Regresión I
	\begin{itemize}
\item Ritchey (2006):
Capítulo 13, p.
464-495
\end{itemize}
\end{enumerate}

\subsection{Sesión 13 (4 de noviembre)}
\begin{enumerate}
\item Correlación y Regresión II
\item Correlación y Regresión en Stata y Excel	
\begin{itemize}
\item Ritchey (2006): Capítulo 15, p.
552-573
\end{itemize}
\end{enumerate}

\subsection{Sesión 14 (11 de noviembre)}
\begin{enumerate}
\item Causalidad
\item ¿Cómo mentir con estadísticas?
\item Entrega taller 3
\begin{itemize}
\item Holland, P. W. (1986) “Statistics and Causal Inference” Journal of
the American Statistical Association 81, 945–970.
\end{itemize}
\end{enumerate}

\subsection{Sesión 15 (18 de noviembre)}
\begin{enumerate}
\item Repaso
\item Entrega Proyecto Final
\begin{itemize}
\item Wendy N. Espeland \& Mitchell L. Stevens (2008): “A Sociology of Quantification”,
European Journal of Sociology.
\end{itemize}
\end{enumerate}

\subsection{Sesión 16 (26 de noviembre)}
\begin{enumerate}
\item {\bf Examen Final}
\end{enumerate}

\section{Nociones Generales sobre el Proyecto Final}
El proyecto de análisis de datos debe tener un máximo de 10 páginas sin anexos. Además, debe estar estructurado siguiendo la línea de artículos científicos:
\begin{itemize}
\item Motivación (¿Por qué es importante el tema?)
\item Marco Teórico
\item Breve Revisión de Literatura
\item Análisis de Datos
\item Conclusiones (Pasos a seguir)

\end{itemize}
\end{document}